\chapter{Introdução}

Nossa vida moderna é extremamente dependente de computadores: desktops, notebooks, netbooks, PDA, celulares, satélites, veículos, microondas, televisores, gps, bancos, energia elétrica, comunicações, etc. Dentro deste contexto, os virus de computador (e suas variações) são uma ameaça real à qual todos, direta ou indiretamente, estamos expostos.

Uma pesquisa feita pela pela Symantec\cite{symantec:7,symantec:8} em 24 países com 13 mil usuários entre 18 e 64 anos, aponta que em 2012 o crime virtual gerou um prejuízo global de mais de 100 bilhões de dólares! Este levantamento, feito anualmente pela Symantec, mostra que existe um grande número de vítimas deste tipo de crime: são aproximadamente 18 vítimas por segundo, ou 556 milhões de vítimas por ano! Ainda segundo o estudo, dois terços da população que usa a internet já foi vítima de algum tipo de crime virtual.

O Brasil aparece em quarto lugar no ranking de países com maior atividade de crimes na internet\cite{uolnoticia:1}, e está entre os que tem maior prejuízo com este tipo de crime. Estima-se que 28 milhões de pessoas foram vítimas em 2012, causando um prejuízo total de 15 bilhões de reais, ou aproximadamente 562 reais por vítima.

Um dos principais fatores que ajudam no aumento destes casos é o acesso cada vez maior da população à internet através de computadores pessoais e, principalmente, smartphones e tablets. As redes sociais, como o facebook, tornaram-se populares entre estes usuários que ainda estão dando os primeiros passos na vida virtual, o que os torna as principais vítimas justamente por não terem usado esta tecnologia anteriormente, nem estarem familiarizados com virus, worms, phishing, etc. Segundo o estudo, dois terços dos usuários não utilizam nenhum sistema de segurança (antivirus, por exemplo) para dispositivos móveis e 44\% destes usuários nem sequer sabe que tais sistemas existem.

Ainda,segundo o estudo:
\begin{itemize}
\item 40\% dos usuários não sabem que um vírus podem passar despercebidos, sem causar nenhum dano aparente ao sistema, o que torna praticamente impossível ao usuário notar que existe o virus na sua máquina. 
\item 49\% concordam que sem o computador parar ou ficar muito lento, não teriam como saber se tem um virus ou malware instalado em seu computador. 
\item 55\% não tem certeza se seu computador está livre de vírus.
\item 3/10 não entendem os riscos virtuais (cybercrimes) e nem sabem como se proteger.
\item 30\% não se preocupam com cybercrime quando estão online, simplesmente porque não acreditam que possam ser vítimas.
\item 21\% não protegem suas informações pessoais quando estão online.
\end{itemize}

Mas ainda temos alguma esperança:
\begin{itemize}
\item 89\% apagam email suspeito de pessoas que não conhecem
\item 83\% possuem ao menos um anti-virus básico
\item 78\% não abrem links ou anexos de e-mails ou textos que eles não solicitaram.
\end{itemize}

O conhecimento é a chave para esclarecer e desmistificar. Esta foi a principal motivação para escrever este trabalho. 
