\chapter{Estrutura de Arquivos PE e ELF}\label{ap:A}

\section{Arquivo PE}

O formato de arquivo PE (Portable Executable Format File) é o último
utilizado para plataforma Microsoft.


\subsection{Estrutura de arquivo PE}\footnote{Melhores informações: http://www.hardware.com.br/tutoriais/executavel-win32/} \footnote{Undocumented PECOFF http://pecoff.reversinglabs.com/}
\begin{list}{}
\item {\begin{tabular}{|l|c|}
\hline 
DOS 2 - Cabeçalho EXE compatível  & \tabularnewline
\cline{1-1} 
Não utilizado  & \tabularnewline
\cline{1-1} 
OEM - Identificador  & Seção DOS 2.0 (para compatibilidade \tabularnewline
OEM - Info  & com DOS somente)\tabularnewline
Offset para cabeçalho PE & \tabularnewline
\cline{1-1} 
DOS 2.0 Stub Program \& Reloc. Table  & \tabularnewline
\hline 
Não utilizado & \tabularnewline
\hline 
PE - Cabeçalho & Palavras limitadas a 8 bytes\tabularnewline
\hline 
Tabela de seções & \tabularnewline
\hline
Image Pages  & \tabularnewline
· Info de Importação & \tabularnewline
· Info de Exportação  & \tabularnewline
· Info de correção & \tabularnewline
· Info de recursos & \tabularnewline
· Info de debug & \tabularnewline
\hline
\end{tabular}}
\end{list}

\subsection{PE - Cabeçalho}


Temos no cabeçalho uma estrutura dividida em campos com palavras
de 4 bytes, enfatizamos alguns deles abaixo:


Tipo de CPU: o campo informa qual o tipo de CPU para a qual o executavel
foi projetado.


Número de Seções: o campo informa o número de entradas na tabela
de seções.


Marca de Tempo/Data: Armazena a data de criação ou modificação do
arquivo.


Flags: Bits para informar qual o tipo de arquivo ou quando há erros
em sua estrutura.


LMAJOR/LMINOR: maior e menor versao do linkador para o executável.


Seção de alinhamento: O valor de alinhamento das seções. Deve ser
múltiplo de 2 dentre 512 e 256M. O valor padrão é 64K.


OS MAJOR/MINOR = Versões limitantes (maior e menor) do sistema operacional.


Tamanho da Imagem: Tamanho virtual da imagem, contando todos os cabeçalhos.
E o tamanho total deve ser multiplo da seção de alinhamento.


Tamanho do Cabeçalho: Tamanho total do cabeçalho. O tamanho combinado
de cabeçalho do DOS, cabeçalho do PE e a tabela de seções.


FILE CHECKSUM: Checksum do arquivo em si, é setado como 0 pelo linkador.


Flags de DLL: Indica qual o tipo de leitura que deve ser feita, processos
de inicialização e terminação de leitura e de threads.


Tamanho reservado da pilha: tamanho de pilha reservado ao programa,
o valor real é o valor efetivo, se o valor reservado não tiver no
sistema ele será paginado.


Tamanho efetivo da pilha: tamanho efetivo.


Tamanho Reservado da HEAP: Tamanho reservado a HEAP.


Tamanho efetivo da HEAP: Valor efetivo para a HEAP.


\subsection{Tabela de Seções}


O número de entradas da tabela de seções e dado pelo campo de número
de seções que está no cabeçalho. A entradas se iniciam em 1. Segue
imediatamente o cabeçalho do PE. A ordem de dados e memória é selecionado
pelo ligador. Os endereços virtuais para s seções são confirmados
pelo ligador de forma crescente e adjacente, e devem sem multiplos
da Seção de alinhamento, que também é fornecida no cabeçalho do PE.
Abaixo alguns de uma seção nesta tabela, divididos em palavras de
8 bytes:


Nome da Seção: Campo com 8 bytes nulos para representar o nome da
seção em ASCII.


Tamanho virtual: O tamanho virtual é o alocado quando a seção é lida.


Tamanho físico: O tamanho de dados inicializado no arquivo para a
seção. É multiplo do campo de alinhamento do arquivo do cabeçalho
do PE e deve ser menor ou igual ao tamanho virtual.


Offset físico: Offset para apntar a primeira página da seção. É relativo
ao inicio do arquivo executavel.


Flags da seção: Flags para sinalizar se a seção é de código, se está
inicializada ou não, se deve ser armazezada, compartilhada, paginável,
de leitura ou para escrita.


\subsection{Páginas de imagem}


A página de imagens contém todos os dados inicializados e todas as
seções. As seções são ordenadas pelo endereço virtual reservado a
elas. o Offset que aponta para a primeira página é especificado na
tabela de seções como visto na subseção acima. Cada seção inicia com
um multiplo da seção de alinhamento.


\subsection{Importação}


A informação de importação inicia com uma tabela de diretórios de
importação que descreve a parte principal da informação de importação.
A tabela de diretórios de importação contém informação de endereços
que são utilizados nas referencias de correção para pontos de entrada
com uma DLL. A tabela de diretórios de importação consiste de um vetor
de entradas de diretórios, uma entrada para cada referencia a DLL.
A última entrada é nula o que indica o fim da tabela de diretórios.


\subsection{Exportação}


A informaçãode exportação inicia com a tablela de diretórios de exportação
que descreve a parte principal da informação de exportação. A tabela
de diretórios de exportação contém informação de endereços que são
utilizados nas referencias de correção para os pontos de entrada desta
imagem.


\subsection{Correção}


A tabela de correção contém todas as entradas de correção da imagem.
O tamanho total de dados de correção no cabeçalho é o número de bytes
na tabela de correção. A tabela de correção é dividida em blocos de
correção. Cada bloco representa as correções para um página de 4K
bytes. Correções que são resolvidados pelo ligador necessitam ser
processadas pelo carregador, a menos que a imagem não possa ser carregada
na Base de imagens especificada no cabeçalho do PE.


\subsection{Recursos}


Recursos são indexados por uma arvore binária ordenada. O design
como um todo pode chegar a $2^{31}$ nivéis, entretanto, NT utiliza
somente 3 niveis: o mais alto com o \emph{tipo}, no subsequente \emph{nome},
depois a \emph{língua}.


\subsection{Debug}


A informação de debug é definido por um debugador que não é controlado
pelo PE ou pelo ligador. Somente é definido pelo PE os dados da tabela
de diretório de debug.


\section{Arquivo ELF}
   O formato ELF(Executable and Linkable Format, anteriormente Extensible Linking Format) é um tipo padrão para formato de arquivos executáveis, códigos, bibliotecas compartilhadas e core dumps. Foi facilmente aceito em diversas distribuições de Unix. 
   Em 1999 foi escolhido como o arquivo binário padrão para o sistemas Unix e baseados em Unix. Diferentemente dos executáveis proprietários ele é flexivel e extensivel, e não está limitado para uma arquitetura especifica. Pode ser adotado por diferentes sistema operacionais e outras plataformas.

\subsection{A estrutura do arquivo ELF}

\begin{list}{}
\item {\begin{tabular}{|l|c|}
\hline
 Arquivo Realocável & \tabularnewline
\cline{1-1}  
Cabeçalho ELF & \tabularnewline
\cline{1-1}
Tabela do cabeçalho do programa (opcional) & \tabularnewline
\cline{1-1} 
seção 1 & \tabularnewline
\cline{1-1} 
seção 2 & \tabularnewline
\cline{1-1} 
... & \tabularnewline
\cline{1-1}
seção n & \tabularnewline
\cline{1-1} 
Tabela de cabeçalho de seção & \tabularnewline
\cline{1-1}
\end{tabular}}
\end{list}

\begin{list}{}
\item {\begin{tabular}{|l|c|}
\hline
Arquivo Carregável &\tabularnewline
\cline{1-1}
Cabeçalho ELF & \tabularnewline
\cline{1-1}
Tabela do cabeçalho do programa & \tabularnewline
\cline{1-1} 
Segmento 1 & \tabularnewline
\cline{1-1} 
 & \tabularnewline
\cline{1-1}
Segmento 2 & \tabularnewline
\cline{1-1}
... & \tabularnewline
\cline{1-1}
Tabela de cabeçalho de seção(opcional) & \tabularnewline
\cline{1-1}
\end{tabular}}
\end{list}


\subsection{Cabeçalho}
   No cabeçalho de um arquivo ELF, existe uma ordem especifica, já para as seções e segmentos não.
   Este contém toda a organização do arquivo, a partir dele que podemos ter acesso a outras partes utilizando o offset.
   Temos as identificações:

   e\_ident: A identificação do arquivo.

   e\_type: Tipo de objeto.

   e\_machine: Arquitetura do arquivo.
   
   e\_version: A sua versão.
  
   e\_entry: Endereço virtual para ponto de inicio do processo.
   
   e\_phoff: O tamanho do cabeçalho em bytes.
   
   e\_phentsize: O tamanho de uma entrada no cabeçalho do ELF.
   
   e\_phnum: O número de entradas no cabeçalho.
   
   e\_flags: São as flags para o processador.
   
   e\_ehsize: O tamanho do header ELF em bytes.
   
   e\_shoff: O tamanho do cabeçalho da seção.
   
   e\_shentsize: O tamanho de uma entrada no cabeçalho da seção.

   e\_shnum- O entradas no cabeçalho da seção.
   
   e\_shstrndx - Indice das seções linkado com a tabela de strings.


\subsection {Identificação}
  Nos 4 bytes iniciais do cabeçalho existe a especificação de como interpreta-lo, não considera o sistema que o le e nem o resto do arquivo.
\subsection {Entry Point Address}
  No e\_entry indica o endereço para onde o sistema irá iniciar a execução dos códigos da seção de texto. Este endereço aponta para o inicio do linkador\(\_start\) e não para o inicio do sistema que o programador define (main).

\subsection {Tabela de Cabeçalhos do Programa (PHT)}
   Descreve a criação do processo para o sistema. É obrigatória para os arquivos executáveis e opcional para os realocáveis. Abaixo segue a descrição de alguns dos identificadores:
   
   p\_type: Tipo de segmento e como interpretar a informação.

   p\_offset: Offset a partir do começo ao primeiro byte do segmento.

   p\_vaddr: Endereço virtual do primeiro byte do segmento.

   p\_paddr: Endereço físico, reservado para quando o é utilizado.

   p\_filesz: Número de bytes do segmento.

   p\_memsz: Número de bytes do segmento na memória.

   p\_flags: Flags utilizadas no segmento.

   p\_align: Valor para alinhamento na memória e no arquivo. Quando 0 ou 1 indica que não é necessário, se o for deve ser positivo e em potência de 2.

\subsection {Tabela de Cabeçalhos de Seção (SHT)}
   Descreve as seções. Cada entrada é definida pela seção e possue informações dela.
   Ela é definida como um vetor, o identificador e\_shoffdo cabeçalho que define o offset para localizar o inicio e o e\_shentsize do tamanho de cada bloco.
   
   sh\_name: Indice para a tabela de string de cabeçalho de seção, descrita acima.

   sh\_type: Tipo semântico da seção.

   sh\_flags: Flags da seção (one-bit, 0 ou 1), que descreve seus atributos.

   sh\_addr: Endereço da seção, caso a seção apareça em memória, senão tem o valor 0.

   sh\_offset: Indica o inicio da seção no arquivo.
   
   sh\_size: Tamanho da seção em bytes.
   
   sh\_link: Contém o link para a tabela de cabeçalho da seção.
   
   sh\_info: Informação extra, interpretado junto com o tipo de seção.
   
   sh\_addralign: Verificação do alinhamento dos blocos definidos por sh\_addr, sh\_addr deve ser divisivel por sh\_addralign.


   sh\_entsize: Tamanho fixo da tabela de seção, se não for deste tipo terá 0.

\subsection {Seções Especiais}
   .bss: Dados não inicializados que são utilizados no programa em memória. Inicia com 0 no inicio do programa, este não ocupa espaço no programa.
   
   .comment: Informações de controle de versão.

   .data ou .data1: Dados inicializados utilizados no programa em memória (tem também a .data1)

   .debug: Informações para debugar os simbolos.
   
   .dynamic: Informações para linkagem dinâmica.
   
   .dynstr: Strings para a linkagem dinâmica.
   
   .dynsym: Tabela de símbolos da linkagem dinâmica.

   .fini: Instruções executáveis para finalização do programa.

   .got: Contém a Global Offset Table \(GOT\).

   .hash: Hashtable de símbolos.

   .init: Instruções para inicialização do programa.

   .interp: Caminho para o programa interpretador.

   .line: Número da linha \(no código\) para debugar.
   
   .note: Informações de formato.
   
   .plt: Contém a Procedure Linkage Tabel \(PLT\).

   .relname ou .relaname: Informações para realocação.

   .rodata ou .rodata1: Dados para somente de leitura, utilizados em segmento que não permite escrita.

   .shstrtab: Nomes das seções.

   .strtab: Nomes associados as entradas na tabela de simbolos.

   .symtab: A tabela de símbolos.

   .text: Instruções executáveis do programa.

\subsection {Tabela de Strings}
   Tabela onde armazena strings, terminadas com '\\0'. O arquivo objeto utiliza-as para representar os simbolos e nomes das seções. É acessada através de seus indices. O campo sh\_name do cabeçalho da seção possue o indice para esta tabela que o é indicado pela e\_shstrndx do cabeçalho do programa.
\subsection {Tabela de Símbolos (Symbol table)}
   Contém referencias e definições para localizar e realocar no programa.
   
   st\_name: Indice para a tabela de strings que contém o nome do símbolo. Se possui 0, o símbolo não tem nome.

   st\_value: Valor do símbolo.

   st\_size: Tamanho do símbolo, caso haja algum senão possui o valor 0.

   st\_info: Tipo do símbolo e atributos.

   st\_other: Possui o valor 0, sem uma definição.
   
   st\_shndx: Indice para o cabeçalho de seção utilizada.
