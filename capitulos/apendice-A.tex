\chapter{Estrutura de Arquivos PE}

\section{Arquivo PE}

O formato de arquivo PE (Portable Executable Format File) é o mais atual para plataforma Microsoft.


\subsection{Estrutura de arquivo PE.}
\begin{list}{}
\item {\begin{tabular}{|l|c|}
\hline 
DOS 2 - Cabeçalho EXE compatível  & \tabularnewline
\cline{1-1} 
Não utilizado  & \tabularnewline
\cline{1-1} 
OEM - Identificador  & Seção DOS 2.0 (para compatibilidade \tabularnewline
OEM - Info  & com DOS somente)\tabularnewline
Offset para cabeçalho PE & \tabularnewline
\cline{1-1} 
DOS 2.0 Stub Program \& Reloc. Table  & \tabularnewline
\hline 
Não utilizado & \tabularnewline
\hline 
PE - Cabeçalho & Palavras limitadas a 8 bytes\tabularnewline
\hline 
Tabela de seções & \tabularnewline
\hline
Image Pages  & \tabularnewline
· Info de Importação & \tabularnewline
· Info de Exportação  & \tabularnewline
· Info de correção & \tabularnewline
· Info de recursos & \tabularnewline
· Info de debug & \tabularnewline
\hline
\end{tabular}}
\end{list}

\subsection{PE - Cabeçalho}


Temos no cabeçalho uma estrutura dividida em campos com palavras
de 4 bytes, enfatizamos alguns deles abaixo:


Tipo de CPU: o campo informa qual o tipo de CPU para a qual o executavel
foi projetado.


Número de Seções: o campo informa o número de entradas na tabela
de seções.


Marca de Tempo/Data: Armazena a data de criação ou modificação do
arquivo.


Flags: Bits para informar qual o tipo de arquivo ou quando há erros
em sua estrutura.


LMAJOR/LMINOR: maior e menor versao do linkador para o executável.


Seção de alinhamento: O valor de alinhamento das seções. Deve ser
múltiplo de 2 dentre 512 e 256M. O valor padrão é 64K.


OS MAJOR/MINOR = Versões limitantes (maior e menor) do sistema operacional.


Tamanho da Imagem: Tamanho virtual da imagem, contando todos os cabeçalhos.
E o tamanho total deve ser multiplo da seção de alinhamento.


Tamanho do Cabeçalho: Tamanho total do cabeçalho. O tamanho combinado
de cabeçalho do DOS, cabeçalho do PE e a tabela de seções.


FILE CHECKSUM: Checksum do arquivo em si, é setado como 0 pelo linkador.


Flags de DLL: Indica qual o tipo de leitura que deve ser feita, processos
de inicialização e terminação de leitura e de threads.


Tamanho reservado da pilha: tamanho de pilha reservado ao programa,
o valor real é o valor efetivo, se o valor reservado não tiver no
sistema ele será paginado.


Tamanho efetivo da pilha: tamanho efetivo.


Tamanho Reservado da HEAP: Tamanho reservado a HEAP.


Tamanho efetivo da HEAP: Valor efetivo para a HEAP.


<<<<<<< HEAD
=======
\subsection{Cabeçalho}

>>>>>>> 156e892e3d131dc73adf19d98edc1e359a8c7b48
Seu cabeçalho descreve partes importantes para o modelo, não é presente ao inicio do arquivo pois está 
 é reservada para compatibilidade com o DOS, que nada mais é que mensagens de aviso que o programa
 não funciona nesta versão.
 São alguns campos importantes para este documento:
 
  
 WORD Machine: Indica a CPU para qual o arquivo foi construido. Diversos virus checam este
 campo para verificar ase a versão é a mesma para a qual foi projetado. Outros virus não o fazem
  e infectam qualquer arquivo o que gera erros para a abertura do arquivo quando é executado em
  uma arquitetura diferente. Este é um risco para virus multiplataforma, poderiam atacar
  diversas arquiteturas com somente um código.

WORD NumberOfSections: São o número de seções em um executável \(DLL\). Este campo tem 
 diversas funções para um virus. Uma delas é que o virus a incrementa para colocar uma 
 parte de si nesta seção. A tabela de seções é alterada no mesmo instante que ocorre essa
 mudança. Sistemas baseados em Windows NT aceitam no máximo 96 seções em um arquivo PE, 
 os baseados em Windows 9x não verificam este campo.

WORD Characteristics: Contém as flags de informação do arquivo. A maioria dos virus
<<<<<<< HEAD
 verificam para ter a certeza de que é um executável, outros ,construidos para infectar
 alguma DLL especifica, para ter certeza que o arquivo é uma DLL. Este campo normalmente
=======
 verificam para ter a certeza de que é um executável, outros, construidos para infectar
 alguma DLL específica, para ter certeza que o arquivo é uma DLL. Este campo normalmente
>>>>>>> 156e892e3d131dc73adf19d98edc1e359a8c7b48
 não é alterado pelos virus.

WORD Magic: O cabeçalho começa inicialmente com um campo "mágico". Os virus o verificam
 para a certeza de que o infectado é um executável comum e não uma ROM ou outro tipo de arquivo.

DWORD SizeOfCode: Este campo tem o teto do tamanho de todas as seções executáveis.
 Geralmente os virus não alteram este valor ao adicionar uma nova seção, mas provavelmente
 isso mude nos virus futuros.

DWORD AddressOfEntryPoint:  É o endereço  de execução que a imagem inicia. Este valor
 normalmente aponta para a seção de texto. Campo crucial para maioria dos virus Win32.
 O campo é alterado pela grande maioria dos tipo de infecções para o ponto de entrado do código malicioso.
 
DWORD ImageBase: Quando o ligador cria p executável, assume que a imagem é mapeada em um local espécifico da memória.
 Este endereço fica armazenado neste campo. Se a imagem pode ser lida de um local especifico, não necessita de realocações pelo carregador.
 Este campo é utilizado por diversos virus para calcular o endereço de alguns itens, mas não é alterado.

DWORD SectionAlignment: quando o executável é mapeado em memória, cada seção inicia de um endereço virtual que é mútiplo deste valor, de no minimo de 4 kb . Grande parte
 dos virus o utilizam para o local correto para se colocar, mas não alteram este campo.

DWORD FileAlignment: Nos arquivos PE, os dados em si iniciam de um multiplo deste valor. Os virus não alteram este valor mas usam de forma similar ao comentado acima \(SectionAlignment\).

DWORD SizeOfImage: Quando o ligador cria a imagem, calcula o tamanho total das partes da imagem que o carregador deve carregar.
 Isso inclui o tamanho da região inicial da imagem base através do fim da última seção. No fim da última seção há o teto do multiplicador do alinhamento da seção.
 Quase todos os arquivos infectados utilizam este valor \(SizeOfImage\). Muitos calculam o campo incorretamente, o que causa a impossibilidade de execução sob Windows NT,
 exatamente pelo motivo de que o Windows 9x não verifica este campo ao executar o arquivo. Para alegria de muitos os criadores de virus não testam muito suas criações. Maior parte
 dos virus de Windows 9x contém esta falha. Muitos antivirus calculam errado este campo na desinfecção, o que causa um executável que era compátivel com windows NT deixe de ser compátivel 
 quando desinfectado.

 DWORD Checksum: Este é o checksum do arquivo. Maior parte dos executáveis registram 0 neste campo. Entretando Dlls e drivers devem tem um checksum diferente de 0. 
  Windows 9x ignora este campo e carrega o arquivo, o que facilita sua infecção no KERNEL32.dll. Este campo é utilizado por muitos virus para representar como marcação, para evitar
  uma dupla infecção, outros tipos de virus recalcula-o para uma infecção mais efetiva.

\subsection{Tabela de Seções}


O número de entradas da tabela de seções e dado pelo campo de número
de seções que está no cabeçalho. A entradas se iniciam em 1. Segue
imediatamente o cabeçalho do PE. A ordem de dados e memória é selecionado
pelo ligador. Os endereços virtuais para s seções são confirmados
pelo ligador de forma crescente e adjacente, e devem sem multiplos
da Seção de alinhamento, que também é fornecida no cabeçalho do PE.
Abaixo alguns de uma seção nesta tabela, divididos em palavras de
8 bytes:


Nome da Seção: Campo com 8 bytes nulos para representar o nome da
seção em ASCII.


Tamanho virtual: O tamanho virtual é o alocado quando a seção é lida.


Tamanho físico: O tamanho de dados inicializado no arquivo para a
seção. É multiplo do campo de alinhamento do arquivo do cabeçalho
do PE e deve ser menor ou igual ao tamanho virtual.


Offset físico: Offset para apntar a primeira página da seção. É relativo
ao inicio do arquivo executavel.


Flags da seção: Flags para sinalizar se a seção é de código, se está
inicializada ou não, se deve ser armazezada, compartilhada, paginável,
de leitura ou para escrita.


\subsection{Páginas de imagem}


A página de imagens contém todos os dados inicializados e todas as
seções. As seções são ordenadas pelo endereço virtual reservado a
elas. o Offset que aponta para a primeira página é especificado na
tabela de seções como visto na subseção acima. Cada seção inicia com
um multiplo da seção de alinhamento.


\subsection{Importação}


A informação de importação inicia com uma tabela de diretórios de
importação que descreve a parte principal da informação de importação.
A tabela de diretórios de importação contém informação de endereços
que são utilizados nas referencias de correção para pontos de entrada
com uma DLL. A tabela de diretórios de importação consiste de um vetor
de entradas de diretórios, uma entrada para cada referencia a DLL.
A última entrada é nula o que indica o fim da tabela de diretórios.


\subsection{Exportação}


A informaçãode exportação inicia com a tablela de diretórios de exportação
que descreve a parte principal da informação de exportação. A tabela
de diretórios de exportação contém informação de endereços que são
utilizados nas referencias de correção para os pontos de entrada desta
imagem.


\subsection{Correção}


A tabela de correção contém todas as entradas de correção da imagem.
O tamanho total de dados de correção no cabeçalho é o número de bytes
na tabela de correção. A tabela de correção é dividida em blocos de
correção. Cada bloco representa as correções para um página de 4K
bytes. Correções que são resolvidados pelo ligador necessitam ser
processadas pelo carregador, a menos que a imagem não possa ser carregada
na Base de imagens especificada no cabeçalho do PE.


\subsection{Recursos}


Recursos são indexados por uma arvore binária ordenada. O design
como um todo pode chegar a $2^{31}$ nivéis, entretanto, NT utiliza
somente 3 niveis: o mais alto com o \emph{tipo}, no subsequente \emph{nome},
depois a \emph{língua}.


\subsection{Debug}


A informação de debug é definido por um debugador que não é controlado
pelo PE ou pelo ligador. Somente é definido pelo PE os dados da tabela
de diretório de debug.



