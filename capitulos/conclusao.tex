\chapter{Conclusão}

Os vírus de computador são um exemplo da capacidade e criatividade humana, a despeito do uso desta criatividade. Os programadores precisam superar muitos desafios, adquirir conhecimentos profundos e ter ideias originais e extremamente criativas.

Os vírus polimórficos foram uma evolução no paradigma de desenvolvimento de algoritmos virais. Representaram e ainda representam um grande desafio para a indústria de proteção contra software mal intencionado. No entanto, existe um ponto fraco que é difícil de ocultar, exigindo cada vez mais criatividade e imaginação por parte dos programadores: a rotina de decodificação do vírus. Apesar de todas as técnicas que discutimos para proteger esta rotina, é através dela que a maioria dos antivírus consegue detectar a ameaça. Mas isso será sempre assim? A história tem mostrado que não. Algum dia alguém criará uma nova técnica e a batalha pelo conhecimento recomeçará! Combater a criatividade de criminosos não é uma tarefa simples, nem fácil. Exige preparação, conhecimento, dedicação e muito interesse pela ciência.

Estamos expostos todos os dias às mais diversas ameaças virtuais, que não podemos ignorar a necessidade e urgência de conhecimento sobre o assunto. Todos os dias milhares de pessoas são vítimas de pessoas mal intencionadas que roubam senhas de banco, números de cartão de crédito, informações confidenciais, segredos industriais. As pessoas pensam somente no prejuízo monetário causado pelo roubo de senhas ou número de cartões de crédito. Mas os vírus também destroem fotos, vídeos, documentos e outras coisas que podem ter um enorme valor emocional ou mesmo representar o trabalho de uma vida. Como mensurar este tipo de prejuízo? Como dizer o quanto vale uma foto da sua formatura, aniversário, casamento ou de uma pessoa querida que faleceu?

Acreditamos que este seja um assunto tão importante que deveria existir uma matéria obrigatória nos cursos de tecnologia sobre virologia e criptografia. Talvez devesse começar como optativa, mas acreditamos ser de teor tão importante que deveria fazer parte da ementa básica de qualquer curso de Ciência da Computação. Há pessoas que acreditam que isso pode ser perigoso. No entanto acreditamos que o conhecimento não seja perigoso, mas sim o que se faz com ele. Os vírus representam um misto de fascínio, curiosidade e medo. O conhecimento e a curiosidade é que movem o ser humano.

John Aycock\footnote{http://pages.cpsc.ucalgary.ca/~aycock/}, professor do Departamento de Ciência da Computação da Universidade de Calgary\footnote{http://www.cpsc.ucalgary.ca/} no Canadá compartilha desta mesma opinião. Ele começou a ensinar em Calgary matérias referentes a vírus e malware em 2003. Já foi muito criticado pelas empresas produtoras de antivírus, mas conta com o apoio de muitas pessoas também. A universidade oferece como disciplina formadora de seus alunos de Ciência da Computação dois cursos sobre ameaças virtuais\footnote{http://www.ucalgary.ca/pubs/calendar/current/computer-science.html\#3658}: \textit{Computer Science 527/627 - Computer Viruses and Malware} e \textit{Computer Science 528/628 - Spam and Spyware}, ambos com 30 horas de duração.

A Universidade de Calgary foi a pioneira, mas não é a única. Existem várias outras iniciativas como esta ao redor do mundo, como por exemplo o \textbf{Bulgarian Academy Of Sciences - National Laboratory of Computer Virology}\footnote{http://www.nlcv.bas.bg/index\_en.htm} e a \textbf{ENSTA Paristech - École Nationale Supérieure de Techniques Avancées} que oferece o curso \textit{IN421 Operational Cryptology and Virology}\footnote{http://wwwdfr.ensta.fr/v2/catalog/course.php?code=IN421\&lang=FR\&lang=EN}

Para concluir, deixamos aqui algumas citações a respeito deste assunto e da natureza humana.
\begin{quotation}
\noindent
\emph{"Viruses don’t harm, ignorance does. Is ignorance a defense?"}\\
herm1t\footnote{herm1t é um programador ucraniando, criador de vírus e colecionador de códigos de vírus. Possui um site onde tem várias informações sobre este tema, incluindo códigos fonte. O site estava disponível em http://vx.netlux.org/index.html mas foi tirado do ar porque o autor foi preso pelas autoridades ucranianas por supostos crimes cibernéticos.}
\end{quotation}

\newpage

\begin{quotation}
\noindent
\emph{"I think computer viruses should count as life ... \\
I think it says something about human nature that the only \\
form of life we have created so far is purely destructive. \\
We've created life in our own image."}\\
Stephen Hawking 
\end{quotation}

\begin{quotation}
\noindent
\emph{"[. . . ] I am convinced that computer viruses are not evil and that\\
programmers have a right to create them, to possess them and to\\
experiment with them . . . truth seekers and wise men have been\\
persecuted by powerful idiots in every age ..."}\\
Mark A. Ludwig\footnote{Mark Ludwig é um físico, autor do livro The Giant Black Book of Computer Virus e pesquisador de vida artificial. É formado pelo MIT, onde concluiu seu curso em apenas 2 anos.}
\end{quotation}

%\begin{quotation}
%\em Todo o indivíduo tem direito à liberdade de opinião e de expressão, \\
%\em o que implica o direito de não ser inquietado pelas suas opiniões \\
%\em e o de procurar, receber e difundir, sem consideração de fronteiras,\\ 
%\em informações e ideias por qualquer meio de expressão.\\
%Artigo 19 da Declaração Universal dos Direitos Humanos
%\end{quotation}

\section{Sugestões para trabalhos futuros}
\begin{itemize}
\item Estudo de técnicas e criação de algorítmo para identificar vírus polimórficos
\item Estudo de técnicas e criação de algorítmo para tornar o executável autoimune
\item Análise dos vírus metamórficos
\end{itemize}
