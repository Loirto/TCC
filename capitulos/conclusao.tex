\chapter{Conclusão}

Os vírus de computador são um exemplo da capacidade e criatividade humana, a despeito do uso desta criatividade. Os programadores precisam superar muitos desafios, adquirir conhecimentos profundos e ter ideias originais e extremamente criativas.

Os vírus polimórficos foram uma evolução no paradigma de desenvolvimento de algoritmos virais. Representaram e ainda representam um grande desafio para a indústria de proteção contra software mal intencionado. No entanto, existe um ponto fraco que é difícil de ocultar, exigindo cada vez mais criatividade e imaginação por parte dos programadores.

Estamos expostos todos os dias às mais diversas ameaças eletrônicas que não podemos ignorar a necessidade e urgência de conhecimento sobre o assunto. Acreditamos que este seja um assunto tão importante que deveria existir uma matéria obrigatória nos cursos de tecnologia sobre virologia e criptografia. Talvez devesse começar como optativa, mas acreditamos ser de teor tão importante que deveria fazer parte da ementa básica de qualquer curso de Ciência da Computação.

\section{Sugestões para trabalhos futuros}
\begin{itemize}
\item Criação de código para identificar vírus polimórficos
\item Criação de código para tornar o executável autoimune
\item Análise dos virus metamórficos
\end{itemize}
