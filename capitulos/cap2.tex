\chapter{Polimorfismo}

\section{O que é o polimorfismo?}

	Segundo o dicionário Aurélio\footnote{http://www.dicionariodoaurelio.com/}\\
\textbf{Polimorfismo}. s.m. Qualidade do que é polimorfo.\\
\textbf{Polimorfo}. adj. Que é sujeito a mudar de forma. Que se apresenta sob diversas formas.


O polimorfismo está diretamente ligado à criptografia. O principal objetivo de um código polimórfico é que duas versões não tenham nada em comum visualmente, evitando assim que o conteúdo seja facilmente identificável e associado à sua verdadeira função. 

\subsection{Polimorfismo usado de forma legal}
Quando se ouve falar em código polimórfico, a primeira associação que se faz é com vírus de computador. No entanto, existem diversas aplicações legais do código polimórfico.

Empresas que desenvolvem código de proteção contra cópia ilegal de software usam código polimórfico para dificultar a engenharia reversa do software de proteção e do software protegido visando dificultar o trabalho de crackers que criam patches para fazer com que o software pense que está registrado legalmente\footnote{Exemplos: http://www.webtoolmaster.com/packer.htm e http://www.pelock.com/products/pelock}. Atualmente, todas as empresas de proteção contra cópia usam polimorfismo juntamente com compactação de dados e técnicas anti-debugging para proteger o software. Alguns usam também códigos metamórficos. O código metamórfico difere do polimórfico por usar técnicas de recompilação/reconstrução fazendo com que cada versão seja única não somente na aparência mas também no código binário executável.

Outro uso legal do código polimórfico seria para gerar uma marca digital do software, pois cada versão seria única tornando possível assim dizer quem é o dono da versão que está em circulação no caso de haver alguma cópia ilegal. A indústria fonográfica já usa algo semelhante a isso nas músicas\footnote{http://en.wikipedia.org/wiki/Digital\_watermarking} com a finalidade de descobrir quem disponibilizou cópias ilegais. O mais importante é que esta medida não altera em nada a qualidade do áudio, passando totalmente despercebida pelo usuário. Também é usado este artifício em máquinas fotográficas e dispositivos de gravação. Cada dispositivo possui sua marca digital única fazendo assim sua 'assinatura' em cada foto ou filme produzidos pelo dispositivo.

\subsection{Como funciona?}

Um código polimórfico exige no mínimo duas partes: a rotina de encriptação e a rotina de decriptação. A criptografia pode ser algo super simples ou algo muito complexo.

Exemplo: 

A função lógica $ \oplus $ (ou exclusivo) faz a combinação binária

\begin{tabular}{|c|c|c|c|}
\hline 
A&B&C = A$ \oplus $B&C$ \oplus $B\\
\hline
0&0&0&0\\
0&1&1&0\\
1&0&1&1\\
1&1&0&1\\
\hline
\end{tabular}

Neste exemplo, A representa o valor de 8, 16, 32 ou 64 bits a ser codificado e B representa a chave de codificação. C é o resultado após a codificação e a última coluna mostra que se aplicarmos o valor codificado à chave original, obtemos novamente o valor original. A função $ \oplus $ é largamente utilizada devido a esta propriedade de facilmente restaurar o valor original.

Apesar desta facilidade, um código feito com esta codificação simplista pode ser facilmente detectado através de análise de padrões. A seguir, vamos demostrar o uso da codificação via operação ou exclusivo, usando o código abaixo.

{{{
\renewcommand{\baselinestretch}{1.0}
\begin{verbatim}
#include <stdlib.h>
#include <stdio.h>
#include <string.h>

int main(void) {
  char  texto[255];
  char  codificado[255];
  int   i, chave;

  printf("Digite o texto a ser codificado (máx. 255 caracteres)\n");
  gets(texto);
  printf("Digite a chave [1..255] para codificar o texto: ");
  scanf("%d", &chave);
  printf("Codificando...\n");
  for (i=0; i < strlen(texto); i++) 
    codificado[i] = texto[i] ^ (char) chave;
  printf("Texto codificado: [%s]\n", codificado);
  printf("Decodificando...\n");
  for (i=0; i< strlen(codificado); i++) 
    codificado[i] = codificado[i] ^ (char) chave;
  printf("Texto decodificado: [%s]\n", codificado);
  return 0;
}
\end{verbatim}
}}}

Vamos usar este código para fazer alguns exemplos


\begin{tabular}{|l|r|l|}
\hline
Texto original&Chave&Texto codificado\\
\hline
A vida passa depressa!&1&@!whe`!q`rr`!edqsdrr` \\
\hline
A vida passa depressa!&23&V7a~sv7gvddv7srgerddv6\\
\hline
A vida passa depressa!&24&Y8nq|y8hykky8|\}hj\}kky9\\
\hline
ABCDEFGHIJKL&13&LONIHKJEDGFA\\
\hline
abcdefghijkl&13&lonihkjedgfa\\
\hline
abc,abc,abc&31&lon!lon!lon\\
\hline
aaaaaaaaaa&54&WWWWWWWWWW\\
\hline
\end{tabular}

Ao olharmos para as três primeiras linhas, a codificação parece ser promissora e parece ser bem difícil, sem ter conhecimento do texto original, e nem da chave, adivinhar o que está escrito no texto codificado. Mas basta um olhar mais atento às 4 últimas linhas e veremos claramente o problema: a formação de padrões.

Então, se há formação de padrões, a codificação torna-se inútil pois um analisador de padrões, ou mesmo uma pessoa, através de análise dos dados e dos padrões de repetição da língua poderiam facilmente encontrar o texto original. No nosso exemplo usamos um texto mas o mesmo princípio aplica-se ao código binário executável e às instruções do processador. Existe um número finito de instruções e suas combinações são bastante conhecidas e portanto descobrir a codificação, ainda que seja uma tarefa trabalhosa, é perfeitamente possível.

Para evitar a formação de padrões, existem várias técnicas que podem ser aplicadas:
\begin{itemize}
\item Rotacionar bits da palavra codificada. Uma sequência de bits tem um bit mais significante e um menos significante. Um byte, por exemplo, pode ser descrito como 76543210, onde 7 é a posição do bit mais significativo e 0 é a posição do bit menos significativo. A rotação de bits pode ser simples: 07654321. Neste caso, o byte foi rotacionado um bit para a direita, ficando o bit menos significativo na posição do mais significativo. Pode parecer uma coisa bem simples, mas usado em combinação com a operação ou exclusivo, esta é uma técnica que dificulta bastante a decodificação. Principalmente porque pode ter rotação variável. Exemplo: os bits  podem rotacionar de acordo com sua posição, ou seja os bytes de posição par podem rotacionar 2 bits e os da  posição ímpar um bit. Ou ainda: Múltiplos de 3, não rotaciona; ímpares rotacionam um para a esquerda e pares 2 rotacionam 1 para a direita. Enfim, o limite da combinação é a imaginação do criador. 
\item Troca de posição dos bytes. Embora mais trabalhosa, é uma técnica interessante. Imaginemos um palavra de 32 bits, onde cada byte é representado pelas letras ABCD, sendo que o byte mais significante o A e o menos significante o D. É possível apenas rotacionando os bits formar BADC ou ainda ACDB, ou qualquer outra combinação desejada. Para dificultar mais a vida de quem vai analisar, pode-se fazer isso de forma não convencional, por exemplo usando grupos de 3 bytes, em vez de 4.
\item Somar a palavra atual com a anterior já codificada, ou aplicar a operação ou exclusivo com a palavra anterior. Neste caso a codificação acontece do início para o fim e a decodificação acontece do fim para o começo. No caso da soma ou subtração tem que tomar cuidado extra por causa do overflow ou underflow. Por exemplo, se somar 20 ao byte de valor 250 o resultado seria 270, que não é possível de representar em 8 bits. No caso da soma a técnica usada é trabalhar usando módulo. No caso, 270 \% 256 = 14. Na verdade o overflow/underflow são desejados pois ajudam a camuflar os resultados. 
\item Embaralhar a ordem da informação. Exemplo: depois de codificado, trocar os itens de posição par com os de posição ímpar.
\item Inserção de lixo no meio dos dados reais. Exemplo: a cada 50 bytes codificados, é inserido 7 bytes de lixo. Este lixo obviamente não tem significado algum e será ignorado na decodificação. No entanto, atrapalha em muito quem está tentando decifrar o código, uma vez que não tem como saber se os dados fazem ou não parte do código real.
\item Utilização de chaves de tamanho variado. A chave pode ser uma sequência de números aleatórios, de tamanho aleatório. Esta sequência aleatória geralmente não é aleatória mas sim pseudoaleatória\footnote{http://pt.wikipedia.org/w/index.php?title=Sequ\%C3\%AAncia\_pseudoaleat\%C3\%B3ria\&oldid=30480084}. Ou é usada uma tabela pré-definida ou uma função que gera a sequência à partir da chave inicial. Neste caso, a grande dificuldade é descobrir a tabela ou a função que gera a chave.
\end{itemize}

Vamos modificar a nossa rotina para usar a operação ou exclusivo juntamente com rotação de bits para vermos se conseguimos nos livrar da formação de padrões. O código modificado está listado abaixo.

{{{
\renewcommand{\baselinestretch}{1.0}
\begin{verbatim}
#include <stdlib.h>
#include <stdio.h>
#include <string.h>

unsigned char rol(unsigned char c, int bits)
{
  return ((c << bits) & 255) | (c >> (8 - bits)); 
}

unsigned char ror(unsigned char c, int bits)
{
  return (c >> bits) | ((c << (8 - bits)) & 255);
}

void imprime_hex(char str[], int tam) {
  int i;
  
  for (i=0; i < tam; i++)
    printf("%02X", (unsigned char) str[i]);
  printf("\n");
}

int main(void) {
  char  texto[255];
  char  codificado[255];
  int   i, c, shift, chave;

  printf("Digite o texto a ser codificado: ");
  gets(texto);
  printf("Digite a chave [1..255] para codificar o texto: ");
  scanf("%d", &chave);
  printf("Codificando...\n");
  shift = 3;
  /* codificação */
  for (i=0; i < strlen(texto); i++) {
    /* aplica ou exclusivo */
    c = (unsigned char) texto[i] ^ chave;
    /* rotaciona os bits */
    if (i % 2 == 0)
      c = ror(c, shift);
    else
      c = rol(c, shift); 
    /* guarda o byte codificado */
    codificado[i] = (char) c;
    shift = (++shift % 3) + 1;
  }
  printf("Devido ao fato da codificação gerar caracteres não ASCII,\n");
  printf("o texto codificado será apresentado em hexadecimal:\n");
  imprime_hex(codificado, strlen(texto));
  printf("Decodificando...\n");
  shift = 3;
  /* decodificação */
  for (i=0; i < strlen(texto); i++) {
    c = codificado[i];
    /* rotaciona os bits */
    if (i % 2 == 0)
      c = rol(c, shift);
    else
      c = ror(c, shift); 
    /* aplica ou exclusivo */
    codificado[i] = (char) c ^ chave;
    shift = (++shift % 3) + 1;
  }
  printf("Texto decodificado: [%s]\n", codificado);
  return 0;

}
\end{verbatim}
}}}

O código mudou e agora a saída não pode mais ser em texto por conta de que certamente teremos muitos caracteres não ASCII. Optamos por mostrar o resultado em hexadecimal. A mudança em relação ao primeiro código que implementamos está apenas na utilização de rotação de bits. Fizemos um código simples apenas para ser didático e mostrar o funcionamento na prática. Usamos um artifício de que quando a posição que estamos codificando for par fazemos a rotação à esquerda. Quando for ímpar, fazemos a rotação à direita. Também variamos o número de bits rotacionados de 1 a 3, conforme vamos avançando na codificação. Novamente frisamos que trata-se de um código simples, sem pretensão de ser perfeito nem otimizado. Codificações de criptografia forte com análise estatística de padrões podem ser encontradas em livros que tratam do assunto de criptografia.

Vamos repetir novamente o estudo da tabela anterior, usando este novo código.

\begin{tabular}{|l|r|p{8cm}|}
\hline
Texto original&Chave&Texto codificado\\
\hline
A vida passa depressa!&1&08 84 BB 43 59 C0 24 C5 30 93 9C C0 24 95 32 8B DC C8 4E C9 30 01\\
\hline
A vida passa depressa!&23&CA DC B0 F3 DC EC E6 9D 3B 23 19 EC E6 CD 39 3B 59 E4 8C 91 3B B1\\
\hline
A vida passa depressa!&24&2B E0 37 8B 1F F2 07 A1 BC 5B DA F2 07 F1 BE 43 9A FA 6D AD BC C9\\
\hline
ABCDEFGHIJKL&13&89 3D 27 4A 12 96 49 15 22 3A 91 82\\
\hline
abcdefghijkl&13&8D BD 37 4B 1A D6 4D 95 32 3B 99 C2\\
\hline
abc,abc,abc&31&CF F5 3E 99 9F FA 8F CC 3F EB 1F\\
\hline
aaaaaaaaaa&54&EA 5D AB BA D5 AE EA 5D AB BA\\
\hline
\end{tabular}

Note que apesar de melhorar muito, na última linha ainda conseguimos ver perfeitamente um padrão. No entanto, quem olha para o padrão não poderá facilmente imaginar que este padrão é um único caracter. Para ficar mais claro o padrão, codificamos novamente a última linha com 20 caracteres e obtivemos:
\begin{center}
\textbf{EA 5D AB BA D5 AE} EA 5D AB BA D5 AE \textbf{EA 5D AB BA D5 AE} EA 5D\\
\end{center}

Pode-se ver claramente o padrão agora que separamos as parte de uma cadeia mais longa. Entretanto, este padrão pode ser diminuido e talvez até eliminado somando-se ao caracter a posição dele na linha em módulo 255 e fazendo a operação ou exclusivo com o caracter codificado anteriormente. O padrão ainda poderá repetir-se quando a codificação resultar em 00. No entanto, já geramos muitos dígitos diferentes para dificultar bastante a decodificação. Não esquecendo que este  padrão só é perceptível por conta de que estamos codificando sempre o mesmo caractere N vezes, o que não ocorre em uma situação real.

A título de curiosidade, o resultado após implementar a soma posicional e a operação ou exclusivo com o caracter anterior pode ser visto abaixo. Note que o único caracter que continua igual é o primeiro pois ele não tem nenhum anterior para aplicar a operação e a posição dele é 0.
\begin{center}
EA B4 19 A4 7D CE 3E 5A E9 2A F5 4C BA D0 69 A0 45 FA 06 76
\end{center}
Caso não tenha entendido o que aconteceu, o segundo caractere gerado foi \textbf{5D} na codificação anterior. Na nova codificação, foi aplicado \textbf{5D + 1 = 5E} em seguida \textbf{5E$ \oplus $EA} o que resultou em \textbf{B4}. O próximo caracter gerado era \textbf{AB}. A este caracter foi aplicado \textbf{AB + 2 = AD} em seguida \textbf{AD$ \oplus $B4}, o que resultou em \textbf{19} e assim sucessivamente. Note que esta codificação é muito mais poderosa pois é feita baseada no caracter anterior e na posição física do caracter. Além disso, acrescentamos uma dificuldade maior, imposta pelo próprio algorítimo, que é a decodificação do final para o começo.

Agora que entendemos como é feita a criptografia, entender o código polimórfico torna-se muito simples. Conforme dissemos anteriormente, é necessário uma rotina de codificação e outra de decodificação. Sem tais rotinas não temos como desenvolver o processo. Em geral a rotina de codificação está em outro módulo - no caso dos software de proteção anti-cópia - ou é codificada junto como restante do código a ser protegido - no caso dos vírus de computador.

Logo de início a primeira pergunta que vem à mente é: \textbf{Mas se existe uma rotina de decodificação, de que adianta o resto estar codificado? Não basta apenas executar a rotina para obter o código original novamente?} De fato, de posse da rotina de faz a decodificação do código criptografado seria muito simples obter o código original. No entanto, existem várias técnicas para impedir o acesso à rotina de decodificação. Este é um dos assuntos do próximo capítulo.

