% APENDICE B
\chapter{As 10 piores pragas virtuais de todos os tempos}

\subsection{Morris - 1988}

	Praga do tipo Worm, lançada em 1988 por Robert Morris. Causou um prejuizo entre 10 e 100 milhões de dolares, não era um worm mal intencionado. Foi criado com o intuito de medir o tamanho da internet, mas tinha uma grande falha ele infectava o mesmo computador diversas vezes, causando DOS (Denial of Service).

\subsection{Melissa - 1999}
	Virus Criado com o nome de uma dançarina da Flórida, a qual o criador era apaixonado (David L. Smith).Causou um prejuizo de 1 bilhão de dólares. Desligava todos os sistemas de e-mails por onde os e-mails infectados passassem. Foi utilizado inicialmente como arquivos de texto que continham senhas de sites pornográficos.

\subsection{Code Red}

	O Code red é um worm que utilizava uma vulnerabilidade de overflow do buffer dos servidores Microsoft IIS e se replicavam para outros servidores de mesmo tipo. Prejuizo de 2 bilhões de dólares. Quando ocorria o estouro de buffer o servidor se desligava o todos os sites armazenados e começavam a exibir a mensagem "Hacked by Chinese!".

\subsection{CIH - 1998}

	Também conhecido como Chernobyl e criado pelo Chen Ing Hau, causou um prejuizo entre 20 e 80 milhões de dólares. Foi um dos mais devastadores, pois além de se reproduzir ele destruia todos os dados do computador, algumas vezes até dados da BIOS, transformando os computador em um peso de papel. Seu poder de propagação foi neutralizado por uma atualização da Microsoft, pois atacava somente versões antigas do Windows (9x e Millenium).

\subsection{Slammer - 2003}

	Worm que causa a ausência de internet na Coreia do Sul por 12 horas. Assim como o Code Red se aproveita da vulnerabilidade do estouro do buffer, mas agora no Microsoft SQL Server. Após a infecção ele causava Denial of Service fazendo com que os banco de dados não respondessem e causasse uma grande lentidão na internet e se replicava a todos os servidores SQL que tivessem a mesma vulnerabilidade, com efeito cascata os sistemas passavam a não responder.

\subsection{Nimda - 2001}
	Não foi determinado qual o prejuizo causado por este worm, havima diversos metodos para espalha-lo, por e-mail, rede, sites e backdoors feitos por outros virus, causou muita lentidão na internet. Por conseguir se espalhar ele  foi considerado o worm mais rápido até o momento, em apenas 22 minutos se tornou o vírus mais espalhado do mundo.

\subsection{Blaster - 2003}
	Criado pelo grupo hacker Xfocus causou o prejuizo de 2 a 10 bilhões de dólares. Com a intenção de atacar sistemas microsoft Windows, se espalhava com a seguinte mensagem "Billy Gates why do you make this possible? Stop Making money and fix your software!!".
\subsection{Sasser - 2004}
	Com prejuizo de 10 milhões de dólares, o worm criado por Sven Jaschan atacou máquinas windows com uma vulnerabildiade de segurança em uma porta de rede que permitia a conexão com outros computadores e se espalhar pela internet, este não foi o motivo de sua fama. Afetou diretamente diversas empresas, como Delta Airlines, quer por se infectar com o virus teve que interromper os vôos, a guarda costeira da Inglaterra teve seus serviçoes de mapas interrompidos e a agência France-Press que também teve comunicações com satélites interrompidas.
\subsection{Storm - 2007}
	Este worm usava o modo de propação de e-mails polêmicos ou sensacionalistas como "Genocídio de muçulmanos britânicos" ou "Fidel Castro faleceu". Já possuia recursos melhores que seus antecessores, o Storm contruiu uma verdadeira "botnet", utilizava os computadores infectados para realizar ações programadas pelo worm, como ataques a sites especificos. Havia comunicações entre as máquinas infectadas para melhorar as formas de ataque.

\subsection{I Love You - 2000}	
	Foi o vírus que trouxe problemas e prejuizos ao redor do mundo, causou um prejuizo entre 5,5 a 8,7 bilhões dólares.  Pelo simples motivo do assunto do e-mail ser "eu te amo" foi muito difundido. Em maio de 2000, por volta de 50 milhões de computadores foram infectados, incluidno órgão dos govenos de todo o mundo. Vários deles, como a CIA, tiveram que desligar o sistema de e-mails para diminuir a propagação dele.