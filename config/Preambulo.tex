%% PREAMBULO do arquivo

\documentclass[12pt,a4paper,brazil]{report}
\usepackage[brazil]{babel}                  % Fornece suporte para os termos na língua portuguesa do Brasil
\usepackage[dvipdfm]{graphicx}              % Para inclusão de figuras (png, jpg, gif, bmp)
\usepackage{graphicx,geometry,color,psfrag} % inclui figuras e trabalha com psfrag
\usepackage{epsfig}                         % Para inclusão de figuras (eps)
\usepackage[utf8x]{inputenc}                % Fornece suporte para caracteres especiais como acentos e cedilha
\usepackage[T1]{fontenc}                    % Fontes para pdf de alta qualidade
\usepackage{ae,aecompl}                     % Fontes para pdf de alta qualidade
\usepackage{amsfonts}                       % Fontes da AMS
\usepackage{amsmath}                        % Comandos matemáticos como /int e multiline
\usepackage{amssymb,bbold}                  % Símbolos matemáticos providos pela AMS
\usepackage{calc}                           % Permite realizar cáculos simples no comando \setlenght
%\usepackage{cite}                          % Agrupar referências (conflita com natbib e hyperref)
\usepackage{fancyhdr}                       % Cabeçalhos
\usepackage{graphics}                       % figuras gráficas
\usepackage{indentfirst}                    % Indenta os primeiros parágrafos
\usepackage{latexsym}                       % Fonte symbol do LkeATEX
\usepackage{makeidx}                        % Índice remissivo
\usepackage{setspace}                       % Para a distância entre linhas
\usepackage{tabularx}                       % Tabelas
\usepackage{multirow}                       % Tabela com linhas divididas 
\usepackage{float}                          % Posicionamento de figuras = opção H 
%\usepackage{hyperref}                      % Insere links em dvi e pdf
\usepackage{doipubmed}                      % inserção de links doi nas referências.     
\usepackage[square,numbers,sort&compress]{natbib}    % Agrupar referências
\usepackage{hypernat}                       % Trabalha junto do hyperref
\usepackage[Lenny]{fncychap}                % Capítulos estilizados.
\usepackage{psfrag}                         % Substituir strings em eps.
%\usepackage{epstopdf}                      % Converte eps para pdf se necessário.
\usepackage{tocbibind}                      % Insere Sumário, referências no tableofcontents.
\usepackage{color}                          % Texto colorido
\usepackage{url}                            % Links
\usepackage[toc,nonumberlist, footnote, acronym]{glossaries}                     % Glossário
\usepackage{appendix}                       % Controle do apêndice
\usepackage{ifpdf}                          % to use same .tex file for both latex & pdflatex
%----------------------------------------------------------------
\usepackage{type1cm}                        %
\usepackage{eso-pic}                        % Para inserir a marca d'água
\usepackage{color}                          %
\usepackage{everypage}                      %
%----------------------------------------------------------------
% The following specifies different options to hyperref depending on
% whether latex or pdflatex is being run.
\ifpdf
  \usepackage[colorlinks,linkcolor=black,urlcolor=black,citecolor=black,
  plainpages=false,pdfpagelabels,breaklinks]{hyperref}
\else
  \usepackage[colorlinks,linkcolor=black,urlcolor=black,citecolor=black,
  plainpages=false,pdfpagelabels,linktocpage]{hyperref}
\fi

%%
%% Algumas definições globais
%%............................................................................

%% Fonte "Arial"
\renewcommand{\rmdefault}{phv} 
\renewcommand{\sfdefault}{phv} 

%% Diretório de imagens
\graphicspath{{img/}}

%% Redefinindo "labels" de tabelas e figuras
\renewcommand{\figurename}{\small\sl Figura}
\renewcommand{\tablename}{\small\sl Tabela}

%% Página de glossário
\newcommand{\glossario}[2]{%
  \newglossaryentry{#1}{name=#1,description={#2}}%
  \glslink{#1}{}%
}
\makeindex %
\makeglossaries %

%% Página de apêndice
\let\plainappendixpage\appendixpage
\makeatletter
\renewcommand{\appendixpage}{%
  \begingroup
  \let\ps@plain\ps@empty
  \plainappendixpage
  \endgroup}
\makeatother
\renewcommand*\appendixpagename{\Large Apêndices}

% Definição do cabeçalho
\newcommand{\CabecalhoInfoCD}{
\begin{center} 
  \begin{tabular*}{0.99\textwidth}%{14cm}
    {@{\extracolsep{\fill}}ll}
    \multirow{3}{*}{\hspace{0.2cm} \includegraphics[width=4.5cm]{../img/Logo.eps}}
    & {\Large \NomeInstituicao} \\
    & {\Large \NomeSetor} \\
    & {\Large \NomeDepartamento} \\
  \end{tabular*}
\end{center} 
}

